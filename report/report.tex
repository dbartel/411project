%%%% Report for CSCE 411 Project
\documentclass{article}
\usepackage{amsmath}

%%%% Title Page
\title{CSCE 411 Project Report}
\author{Daniel Bartel}
\date{}



\begin{document}
  \maketitle
  \section*{Problem Description}
  The chosen problem was the single-source shortest path problem. Given a weighted graph, the shortest path algorithm will compute the shortest paths from a given node to every other node in the graph.
\\ \ \\
Shortest path algorithms can have many real world applicatoins including:
\begin{enumerate}
  \item[\textbullet] Driving Directions
  \item[\textbullet] AI Path finding
  \item[\textbullet] Routing data through a network
\end{enumerate}
  \section*{Algorithm Overview}
  The two competing algorithms were Dijkstra's algorithm and the Bellman-Ford algorithm. Both algorithms use the idea of ``relaxing'' edges, which involves calculating approxmiate distances and replacing them with more accurate values until the correct solution is reached.
    \subsection*{Bellman-Ford}
    \subsection*{Dijkstra}
  \section*{Implementation}
The algorithms were implemented in Common Lisp. In addition to the attached source code, the project is available on GitHub.
  \section*{Analysis}
  \section*{Conclusion}
\end{document}
